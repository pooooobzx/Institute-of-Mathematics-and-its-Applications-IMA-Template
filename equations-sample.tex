%%
%% Copyright 2021 OXFORD UNIVERSITY PRESS
%%
%% This file is part of the 'ima-authoring-template Bundle'.
%% ---------------------------------------------
%%
%% It may be distributed under the conditions of the LaTeX Project Public
%% License, either version 1.2 of this license or (at your option) any
%% later version.  The latest version of this license is in
%%    http://www.latex-project.org/lppl.txt
%% and version 1.2 or later is part of all distributions of LaTeX
%% version 1999/12/01 or later.
%%
%% The list of all files belonging to the 'ima-authoring-template Bundle' is
%% given in the file `manifest.txt'.
%%
%% Template article for OXFORD UNIVERSITY PRESS's document class `ima-authoring-template'
%% with bibliographic references
%%

\documentclass[numbers,webpdf,imaiai]{ima-authoring-template}%
%\documentclass[namedate,webpdf]{ima-authoring-template}

%\usepackage{showframe}

\graphicspath{{Fig/}}

% line numbers
%\usepackage[mathlines, switch]{lineno}
%\usepackage[right]{lineno}

\theoremstyle{thmstyleone}%
\newtheorem{theorem}{Theorem}%  meant for continuous numbers
%%\newtheorem{theorem}{Theorem}[section]% meant for sectionwise numbers
%% optional argument [theorem] produces theorem numbering sequence instead of independent numbers for Proposition
\newtheorem{proposition}[theorem]{Proposition}%
%%\newtheorem{proposition}{Proposition}% to get separate numbers for theorem and proposition etc.

\theoremstyle{thmstyletwo}%
\newtheorem{example}{Example}%
\newtheorem{remark}{Remark}%

\theoremstyle{thmstylethree}%
\newtheorem{definition}{Definition}

\numberwithin{equation}{section}

\begin{document}

\DOI{DOI HERE}
\copyrightyear{2021}
\vol{00}
\pubyear{2021}
\access{Advance Access Publication Date: Day Month Year}
\appnotes{Paper}
\copyrightstatement{Published by Oxford University Press on behalf of the Institute of Mathematics and its Applications. All rights reserved.}
\firstpage{1}

%\subtitle{Subject Section}

\title[Math in LaTeX]{Math in LaTeX}


\author{\textsc{First Author*}
\address{\orgdiv{Department}, \orgname{Organization}, \orgaddress{\street{Street}, \postcode{Postcode}, \state{State}, \country{Country}}}}

\authormark{Author Name et al.}

\corresp[*]{Corresponding author. \href{email:email-id.com}{email-id.com}}

\received{Date}{0}{Year}
\revised{Date}{0}{Year}
\accepted{Date}{0}{Year}

%\editor{Associate Editor: Name}

%\abstract{
%\textbf{Motivation:} .\\
%\textbf{Results:} .\\
%\textbf{Availability:} .\\
%\textbf{Contact:} \href{name@bio.com}{name@bio.com}\\
%\textbf{Supplementary information:} Supplementary data are available at \textit{Briefings in Bioinformatics}
%online.}

\abstract{Abstracts must be able to stand alone and so cannot contain citations to
the paper's references, equations, etc. An abstract must consist of a single
paragraph and be concise. Because of online formatting, abstracts must appear
as plain as possible.}

\keywords{keyword1; Keyword2; Keyword3; Keyword4.}

% \boxedtext{
% \begin{itemize}
% \item Key boxed text here.
% \item Key boxed text here.
% \item Key boxed text here.
% \end{itemize}}

\maketitle


\section{Introduction}

There are two major modes for typesetting math in LaTeX --- one is embedding the math directly into your text by encapsulating
your formula either using single dollar signs --- \verb+$...$+ or by enclosing the inline math with the following symbol
\verb+\(...\)+ and the other is using a predefined math environment. This file provide basic samples for inline and display
math equations. However, for detailed explanation, you are requested to refer ``amsmath'' package documentation.

\section{Inline math examples}

LaTeX needs to know when text is mathematical. This is because LaTeX typesets maths $a^{2}_{b}$ notation differently
\([a+b]^{2}\) from normal text. Therefore, special \(\sqrt{x^{2}+y^{2}}\) environments have been declared for this purpose.



\section{Display math examples}

\subsection{Default math environments}

The \verb+\begin{equation}...\end{equation}+ environment is used for numbered single line equations:
\begin{equation}
\mathbf{P} = \lim_{\Delta v\to 0} \varepsilon
\left[\frac{1}{\Delta v}\sum_{i=1}^{N_{e}\Delta v}d\mathbf{p}_{i}\right] =
N_{e}d\mathbf{p}_{\mathrm{av}}
\end{equation}

The \verb+\begin{equation*}...\end{equation*}+ environment is used for unnumbered single line equations:
\begin{equation*}
L_s=\mu_1h=\mu_1=\frac{BW}{\omega_o\sqrt{\mu_2\epsilon_2}}=\frac{\mu_1}{\sqrt{\mu_2\epsilon_2}}
\end{equation*}

\subsection{amsmath package --- math environments}

The \verb+\begin{align}...\end{align}+ environment is used for two or more equations when vertical alignment is desired;
usually binary relations such as equal signs are aligned. To have several equation columns side-by-side, use extra ampersands
to separate the columns. You can suppress the number on any particular line by putting \verb+\notag+ before the end of that line;
\verb+\notag+ should not be used outside a display environment as it will mess up the numbering. You can also override a number
with a tag of your own by using \verb+\tag{<label>}+, where \verb+<label>+ means arbitrary text such as \verb+*+ or \verb+\dag+
used to `number' the equation:
\begin{align}
z &= x+y &z &= x+y \\
  &= z+y &x &= z+y \nonumber\\
a &= b+c &a &= b+c \tag{*} \\
  &= z+y &x &= z+y \notag\\
b &= b+c &a &= b+c \tag{\dag} \\
c &= b+c &a &= b+c \tag{\ddag}
\end{align}

The \verb+\begin{align*}...\end{align*}+ environment is used for unnumbered math environments:
\begin{align*}
z &= x+y &z &= x+y \\
a &= b+c &a &= b+c
\end{align*}

The \verb+gather+ environment is used for a group of consecutive equations when there
is no alignment desired among them; each one is centered separately within
the text width. Equations inside gather are separated by a \verb+\\+ command.
\begin{gather}
(a+b)^{2} = a^{2} + 2ab + b^{2} \nonumber \\
a_{2} = b_{2} + c_{2} - d_{2} + e_{2}
\end{gather}

The \verb+gather*+ environment is used for a group of consecutive unnumbered equations for which no alignment is required.
\begin{gather*}
(a+b)^{2} = a^{2} + 2ab + b^{2} \\
a_{2} = b_{2} + c_{2} - d_{2} + e_{2}
\end{gather*}

A variant environment \verb+\begin{alignat}...\end{alignat}+ allows the horizontal space between equations
to be explicitly specified. This environment takes one argument, the number of ``equation columns'' (the number of
pairs of right-left aligned columns; the argument is the number of pairs): count the maximum number of \verb+&s+ in any row,
add 1 and divide by 2
\begin{alignat}{2}
x &= y_1 - y_2 + y_3 - y_5 + y_8 - \dots  &\quad &\text{by Axiom 1.} \nonumber \\
  &= y' \circ y^{*} &\quad &\text{by Axiom 2.} \nonumber\\
  &= y(0) y' &\quad &\text{by Axiom 3.}
\end{alignat}

The \verb+\begin{alignat*}...\end{alignat*}+ is available for setting unnumbered equation
with the above preferences.
\begin{alignat*}{2}
x &= y_1 - y_2 + y_3 - y_5 + y_8 - \dots  &\quad &\text{by Axiom 1.} \\
  &= y' \circ y^{*} & &\text{by Axiom 2.} \\
  &= y(0) y' & &\text{by Axiom 3.}
\end{alignat*}

The \verb+aligned+ environment inside \verb+equation+ environment can be used to produce the below
equation: \verb+\begin{equation}\begin{aligned}...\end{aligned}\end{equation}+
\begin{equation}
\left.\begin{aligned}
B' &= -\partial\times E,\\
   &= -\partial\times Z,\\
E' &= \partial\times B - 4\pi j,
\end{aligned}\right\} \qquad \text{Maxwell's equations}
\end{equation}

Example for \verb+aligned+ environment inside \verb+equation*+ environment:
\verb+\begin{equation*}+ \verb+\begin{aligned}...\end{aligned}\end{equation*}+
\begin{equation*}
\left.\begin{aligned}
B' &= -\partial\times E,\\
   &= -\partial\times Z,\\
E' &= \partial\times B - 4\pi j,
\end{aligned}\right\} \qquad \text{Maxwell's equations}
\end{equation*}

The \verb+subarray+ environment can be used to produce centered multiline subscript or
superscript --- \verb+\begin{subarray}{c}. . .\end{subarray}+
\begin{equation}
\sum_{\begin{subarray}{c}
i\in\Lambda\\
0<j<n
\end{subarray}} P(i,j) = \partial\times B - 4\pi j,
\end{equation}

Example for \verb+subarray+ with left aligned multiline subscript
inside \verb+equation+ --- \verb+\begin{subarray}{l}...\end{subarray}+
\begin{equation}
\sum_{\begin{subarray}{l}
i\in\Lambda\\
0<j<n
\end{subarray}} P(i,j) = \partial\times B - 4\pi j,
\end{equation}

The environment flalign (``full length alignment'') stretches the space between
the equation columns to the maximum possible width, leaving only enough space
at the margin for the equation number, if present.
\begin{flalign}
a_{11} &=b_{11} &a_{12} &=b_{12} \nonumber\\
a_{21} &=b_{21} &a_{22} &=b_{22} + c_{22} \\
       &=b_{21} &a_{22} &=b_{22} + c_{22}
\end{flalign}

Example for \verb+\begin{flalign*}...\end{flalign*}+ environment.
\begin{flalign*}
a_{11} &=b_{11} &a_{12} &=b_{12}\\
a_{21} &=b_{21} &a_{22} &=b_{22} + c_{22} \\
       &=b_{21} &a_{22} &=b_{22} + c_{22}
\end{flalign*}

The \verb+multline+ environment is a variation of the equation environment used for
equations that don't fit on a single line. The first line of a multline will be at
the left margin and the last line at the right margin. Any additional lines in between
will be centered independently within the display width \verb+\begin{multline}...\end{multline}+
\begin{multline}
a+b+c+d+e+f \\
a+b+c+d+e+f \\
a+b+c+d+e+f \\
+i+j+k+l+m+n
\end{multline}

The \verb+multline*+ environment
\begin{multline*}
a+b+c+d+e+f \\
a+b+c+d+e+f \\
a+b+c+d+e+f \\
+i+j+k+l+m+n
\end{multline*}

For cases like constructions, \verb+\begin{cases}...\end{cases}+ inside \verb+equation+ environment can be used.
\begin{equation}
P_{r-j}=\begin{cases}
0& \text{if $r-j$ is odd},\\
r!\,(-1)^{(r-j)/2}& \text{if $r-j$ is even}.
\end{cases}
\end{equation}

For unnumbered cases like constructions, \verb+\begin{cases}...\end{cases}+ inside \verb+equation*+ environment can be used.
\begin{equation*}
P_{r-j}=\begin{cases}
0& \text{if $r-j$ is odd},\\
r!\,(-1)^{(r-j)/2}& \text{if $r-j$ is even}.
\end{cases}
\end{equation*}

\subsection{Matrices}

The amsmath package provides some environments for matrices beyond the basic array environment of LaTeX.
The \verb+pmatrix+, \verb+bmatrix+, \verb+Bmatrix+, \verb+vmatrix+ and \verb+Vmatrix+ have (respectively)
\verb+( )+, \verb+[ ]+, \verb+{ }+, \verb+| |+, and \verb+|| ||+ delimiters built in. Example for
\verb+\begin{pmatrix}...\end{pmatrix}+ environment
\begin{equation*}
\alpha + \gamma_{2} = \begin{pmatrix}
1 &2-3 &a+b\\
1 &2-3 &c+d\\
1 &2-3 &c+d
\end{pmatrix}
\end{equation*}

Example for \verb+\begin{bmatrix}...\end{bmatrix}+ environment
\begin{equation}
\alpha + \gamma_{2} = \begin{bmatrix}
1 &2-3 &a+b\\
1 &2-3 &c+d\\
1 &2-3 &c+d
\end{bmatrix}
\end{equation}

Example for \verb+\begin{Bmatrix}...\end{Bmatrix}+ environment
\begin{equation*}
\alpha + \gamma_{2} = \begin{Bmatrix}
1 &2-3 &a+b\\
1 &2-3 &c+d\\
  &2-3 &c+d
\end{Bmatrix}
\end{equation*}

Example for \verb+\begin{vmatrix}...\end{vmatrix}+ environment
\begin{equation*}
\alpha + \gamma_{2} = \begin{vmatrix}
1 &2-3 &a+b\\
1 &2-3 &c+d\\
1 &2-3 &c+d
\end{vmatrix}
\end{equation*}

Example for \verb+\begin{Vmatrix}...\end{Vmatrix}+ environment
\begin{equation*}
\alpha + \gamma_{2} = \begin{Vmatrix}
1 &2-3 &a+b\\
1 &2-3 &c+d\\
1 &2-3 &c+d
\end{Vmatrix}
\end{equation*}

%If you need ``border'' or ``indexes'' on your matrix, plain TeX provides the macro \verb+\bordermatrix+
%\[ M = \bordermatrix[{[]}]{  & x & y \cr
%                  A & 1 & 0 \cr
%                  B & 0 & 1 \cr} \qquad \bordermatrix[{\{\}}]{  & x & y \cr
%                  A & 1 & 0 \cr
%                  B & 0 & 1 \cr} \qquad \bordermatrix[{()}]{  & x & y \cr
%                  A & 1 & 0 \cr
%                  B & 0 & 1 \cr}
%\]


The wrapper environment \verb+\begin{subequations}...\end{subequations}+ can be used along with a particular
\verb+align+ or similar groups to produce a subordinate numbereing sequence.
\begin{subequations}
\begin{align}
A_1 &= N_0(\lambda;\Omega')-\phi(\lambda;\Omega'), \nonumber\\
A_2 &= \phi(\lambda;\Omega')-\phi(\lambda;\Omega), \\
A_3 &= \mathcal{N}(\lambda;\omega).
\end{align}
\end{subequations}

The \verb+split+ environment is for single equations that are too long to fit on one line and hence must be split into
multiple lines. The \verb+split+ environment provides for alignment among the split lines, using \verb+&+ to mark alignment
points. Unlike the other amsmath equation structures, the split environment provides no numbering, because it is intended
to be used only inside some other displayed equation structure, usually an \verb+equation+, \verb+align+,
or \verb+gather+ environment, which provides the numbering. For example:
\begin{equation}
\begin{split}
H_{c} &= \frac{1}{2n} \sum^{n}_{l=0}(-1)^{l} (n-{l})^{p-2} \sum_{l_1+\dots+l_p=l} \prod^{p}_{i=1} \binom{n_i}{l _i}\\
&\quad\cdot[(n-l)-(n_i-l_i)]^{n_i - l_i}\cdot \Bigl[(n-l)^2 - \sum^{p}_{j=1}(n_{i} - l_{i})^2\Bigr].
\end{split}
\end{equation}

\subsection{Additional math samples}

Integration with side limits - superscript and subscript:
\[
\int^{A}\qquad \int_{A}\qquad \int_{A}^{B}
\]

Summation with under/over limits:
\[
\sum_{A}\qquad \sum^{z+y}\qquad \sum_{a+b}^{b}
\]

Math with \verb+underline+:
\[\underline{b+c=d}\]

Math with \verb+underbar+:
\[\underbar{b+c+z=y}\]

Math with \verb+underbrace+:
\[\underbrace{a+b=c^2 + y_2 (a^2)^2}\]

\subsection{Over and under arrows}

Basic LaTeX provides \verb+\overrightarrow+ and \verb+\overleftarrow+ commands. Some
additional over and under arrow commands are provided by the amsmath package to extend the set.


Math with \verb+underrightarrow+:
\[\underrightarrow{a+b_c+y}\]

Math with \verb+underleftarrow+:
\[\underleftarrow{a+b_c+y}\]

Math with \verb+underleftrightarrow+:
\[\underleftrightarrow{a+b_c+y}\]

Math with \verb+overline+:
\[\overline{(a+b=c)}\]

Math with \verb+overbrace+:
\[\overbrace{a+b+c}\]

Math with \verb+overrightarrow+:
\[\overrightarrow{a+b+c} \]

Math with \verb+overleftarrow+:
\[\overleftarrow{a + b + c}\]

Math with \verb+overleftrightarrow+:
\[\overleftrightarrow{a + b + c}\]

Math with \verb+\underset{...}+:
\[ = 2\cos (2\cdot \underset{\mbox{frequency}}{\underset{\mbox{average}}{\underbrace{327}}} \pi t) \cos
(\underset{\mbox{second}}{\underset{\mbox{beats per}}{\underbrace{130}}} \pi t)\]

Math with \verb+\overset{...}+:
\begin{equation*}
\left(\frac{BW}{\omega_o}\right)
\overset{\overbrace{a+b}}{=} B
\left(\frac{BW}{\omega_o}\right)
\overset{\overset{a}{b}}{=} B
\left(\frac{BW}{\omega_o}\right)
\overset{\mu_1=\mu_2}{\overbrace{=}} B
\left(\frac{BW}{\omega_o}\right)
\overset{\underbrace{\mu_1=\mu_2}}{=} B
\left(\frac{BW}{\omega_o}\right)
\end{equation*}

Boxed equation: \verb+\begin{equation}\boxed{...}+ \verb+\end{equation}+
\begin{equation}
 \boxed{x^2+y^2 = z^2}
\end{equation}

\subsection{Extensible arrows}

\verb+\xleftarrow+ and \verb+\xrightarrow+ produce arrows that extend automatically to
accommodate unusually wide subscripts or superscripts. These commands take
one optional argument (the subscript) and one mandatory argument (the superscript, possibly empty):
\[
\xleftarrow[a+c]{x^2+2xy+y^2} \qquad
\xrightarrow[a+c]{x^2+2xy+y^2}\qquad
\xleftarrow[x+y^2]{\text{maps to}}\qquad
\xrightarrow[x+y^2]{\text{maps to}}\qquad
\xleftarrow[a+c]{\text{maps to}}
\]

\subsection{Math accents}

\begin{equation*}
\begin{array}{ll@{\hskip36pt}ll}
\verb+\hat{a}+ & \hat{a} & \verb+\bar{a}+ & \bar{a}  \\
\verb+\grave{a}+ & \grave{a} & \verb+\acute{a}+ & \acute{a} \\
\verb+\dot{a}+ & \dot{a} & \verb+\ddot{a}+ & \ddot{a}  \\
\verb+\breve{a}+ & \breve{a} & \verb+\mathring{a}+ & \mathring{a}\\
\verb+\stackrel\frown{a}+ & \stackrel\frown{a} & \verb+\check{a}+ & \check{a}\\
\verb+\vec{a}+ & \vec{a} & \verb+\tilde{a}+ & \tilde{a} \\
%%\verb+\widehat{a+b}+ & $\widehat{a+b}$ & \verb+\widetilde{a+b}+ & $\widetilde{a+b}$ \\
\end{array}
\end{equation*}

\section*{Revision history}

\begin{tabular}{|l|l|l|l|}
\hline
Revision State & Revision Date & Version Number & Revision History \\
\hline
00   & 16 December, 2021 & Version 1.0 & \\
\hline
\end{tabular}

\end{document}




